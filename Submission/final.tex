% !TEX TS-program = pdflatex
% !TEX encoding = UTF-8 Unicode

% This is a simple template for a LaTeX document using the "article" class.
% See "book", "report", "letter" for other types of document.

\documentclass[11pt]{article} % use larger type; default would be 10pt

\usepackage[utf8]{inputenc} % set input encoding (not needed with XeLaTeX)
\usepackage[backend=biber, style=authoryear]{biblatex}
\addbibresource{bibliography.bib}
%%% Examples of Article customizations
% These packages are optional, depending whether you want the features they provide.
% See the LaTeX Companion or other references for full information.

%%% PAGE DIMENSIONS
\usepackage{geometry} % to change the page dimensions
\geometry{a4paper} % or letterpaper (US) or a5paper or....
% \geometry{margin=2in} % for example, change the margins to 2 inches all round
% \geometry{landscape} % set up the page for landscape
%   read geometry.pdf for detailed page layout information

%\usepackage{graphicx} % support the \includegraphics command and options

\usepackage[parfill]{parskip} % Activate to begin paragraphs with an empty line rather than an indent

%%% PACKAGES
\usepackage{caption, booktabs} % for much better looking tables
\captionsetup{  justification = centering }
\usepackage{array} % for better arrays (eg matrices) in maths
\usepackage{paralist} % very flexible & customisable lists (eg. enumerate/itemize, etc.)
\usepackage{verbatim} % adds environment for commenting out blocks of text & for better verbatim
\usepackage{subfig} % make it possible to include more than one captioned figure/table in a single float
\usepackage{graphicx}
\usepackage{rotating}
\usepackage{multirow} % allows multiple rows per cell
% These packages are all incorporated in the memoir class to one degree or another...

%%% HEADERS & FOOTERS
\usepackage{fancyhdr} % This should be set AFTER setting up the page geometry
\pagestyle{fancy} % options: empty , plain , fancy
\renewcommand{\headrulewidth}{0pt} % customise the layout...
\lhead{}\chead{}\rhead{}
\lfoot{}\cfoot{\thepage}\rfoot{}

%%% SECTION TITLE APPEARANCE
\usepackage{sectsty}
\allsectionsfont{\sffamily\mdseries\upshape} % (See the fntguide.pdf for font help)
% (This matches ConTeXt defaults)

%%% ToC (table of contents) APPEARANCE
\usepackage[nottoc,notlof,notlot]{tocbibind} % Put the bibliography in the ToC
\usepackage[titles,subfigure]{tocloft} % Alter the style of the Table of Contents
\renewcommand{\cftsecfont}{\rmfamily\mdseries\upshape}
\renewcommand{\cftsecpagefont}{\rmfamily\mdseries\upshape} % No bold!

%%% END Article customizations

%%% The "real" document content comes below...

\title{\vspace{-2.0cm}UCL Study Space Search Costs}

%\date{} % Activate to display a given date or no date (if empty),
         % otherwise the current date is printed 

\begin{document}
\maketitle

975 words excluding headings, figures, and references. \\

\section{Introduction}

This work uses agent based modeling to study how the structure of study space and the \textbf{UCL Go} study space app affect search costs.

\section{Literature Review}

Reservation systems alter the dynamics of the search for parking space \parencite{TasseronG.2017Upsr}. In that work, a subset of agents shared parking availability information. Search times improved for connected cars but  at a cost to non-connected cars for no net social benefit. The key structural difference between this research  and the work cited is the bottom up granular parking information compared to the top down information from the UCL Go app. That work relied upon an ABM developed by \cite{BenensonItzhak2008PAam} to model driver decisions at intersections and the trade-off between an available spot and the possibility of finding one closer to the final destinations. While this work is helpful, the differences between this and the search for study space are significant.  

\section{Methodology}

An agent based model was built in Netlogo and calibrated to approximate the distance, study space density, and  population of UCL. Table 1 contains the data for travel time between and search time within major UCL study spaces. Search times range from 200 seats per minute in the Student center to 30 per minute in the Bartlett Library/ The diameter of campus is 15 or 20 minutes walking. 

First, a minimum number of runs for parameters like UCL characteristics was found. 

Then effect of scaling the model was considered. Actual parameter values for UCL would be probably 50-75\% of the 40,000 enrolled students searching the roughly 4,000 study spaces across campus in order to do at least 1-2 hours of work. Parameters of this magnitude were unfeasible so the model was scaled down, maintaining accurate ratio's where possible but reducing the number of students, spaces, places, and total work. Thus it was important to compare scales. 

To investigate dispersion's effect search costs, density was changed.

Overall, the model did not allow for a parameter sweep, so the methodology focused on investigating key concerns about drawing conclusions from the model and results are discussed in this context. 

% table of data about campus walking times and library search times

\begin{table}[]
\centering
\setlength\tabcolsep{0pt} % default value: 6pt
\footnotesize
\begin{tabular}{l|c|cccccc}
Location 		& index \space & 1   & 2     				& 3 					& 4 					& 5 					& 6 \\ \hline
Bart. Lib 		& 1 	& \fbox{69 \{2:00\}}& 4:30  				& 8:00*					& 9:01					& 8:30					& 5:00*	\\
Student Ct. 	& 2 	& 4:30				& \fbox{647 \{3:00\}}	& 1:43					& 4:31					& 4:10					& 5:00* \\
Main Lib 		& 3 	& 6:13				& 2:43					& \fbox{645 \{8:02\}}	& 2:48					& 2:27					& 7:00*	\\
Science Lib 	& 4 	& 9:01				& 7:18					& 2:48					& \fbox{ 925 \{ 6:28\}}	& 5:15					& 6:00*	\\
Cruciform 		& 5 	& 8:40				& 5:10					& 2:27					& 5:15					& \fbox{ 326 \{2:08\}}	& 9:00*	\\
IoE 			& 6 	& 5:00*				& 5:00*					& 7:00*					& 6:00*					& 9:00*					& \fbox{ 326 \{5:58\}}	\\
Kings Cross	\space & 7 	& 9:01				& 16:00*       			& 18:00*				& 21:00*				& 18:00*				& 21:00*	\\
Warren St. 		& 8 	& 10:00*			& 8:00*					& 4:00*					& 9:00*					& 3:00*					& 11:00*	\\
Russell Sq 		& 9 	& 7:37				& 10:00*				& 12:00*				& 10:00*				& 13:00*				& 3:00*					
\end{tabular}
\caption{Walking time data for campus locations \\ \footnotesize{Time to walk between locations. Where $i = j$ notation is'' ``total seats \{search time\} '' 
\\ * indicates walking time taken from Google Maps}}
\end{table}

\section{Results}

Results are found in tables 2-5. The key value in each table is the number of ticks required for all work to be ``completed.'' ``Efficiency'' is calculated, which is:  total work completed, / number of turtles X total ticks. This has the value of adjusting for differing total amounts of work due to the random initialization of work amounts. $Function$ indicates how agents searched for space. $Most$, agents searched by absolute value of spaces. $Prox$, agents searched by total distance to the patch. $Perc$ agents searched by percentage of seats available. The ODD and code includes implementation specifics. Lastly, trial values from one run are never included in the values of another run, so that the sample statistics are not biased towards each other. This is important for interpreting tables 2, 3, and 4, where  minimum required runs is considered. 

% table of trial data

\begin{table}[]
\footnotesize
\centering
\begin{tabular}{l|l|ccc|ccc}
 & Trial & 1 & 2 & 3 & 4 & 5 & 6  \\ \hline
  \parbox[t]{2mm}{\multirow{7}{*}{\rotatebox[origin=c]{90}{parameters}}} 
 & step size 	& 10  			& 10  			& 10 			& 10 					& 10 				& 10  \\
 & work avg 	& 60 			&  60 			& 60			& 60 					& 60 				& 60 \\
 & students 	& 1000 			& 1000 			& 1000 			& 2000 					& 2000 				& 2000 \\
 & places 		& 5 			& 5 			& 5 			& 10 					& 10 				& 10 \\
 & spaces 		& 250 			& 250 			& 250 			& 500 					& 500 				& 500 \\
 & function 	& most 			& prox 			& perc 			& most 					& prox 				& perc \\
 & trials 		& 5 			& 5 			& 5 			& 5 					& 5 				& 5 \\ \hline
\parbox[t]{2mm}{\multirow{3}{*}{\rotatebox[origin=c]{90}{results}}} 
 & ticks 		& $164\pm 8.6$ 	& $143\pm 5.8$ 	& $151.2\pm 5.7$	& $155.4\pm 4.0$	& $136.8\pm 6.1$	& $150.4\pm 2.1$ \\
 & efficiency 	& 36\%			& 42\%			& 40\%				& 39\%				& 44\%				& 40\% \\
 & runtime (s) 	& 52.49			& 43.27			& 48.90				& 115.56			& 100.38			& 118.90
\end{tabular}
\caption{Parameters and results of various trial sets discussed \\ \footnotesize In results, tick value is ``$mean \pm margin \: of \: error$''}
\end{table}

%redo with 10 trials each

\begin{table}[]
\footnotesize
\centering
\begin{tabular}{l|l|ccc|ccc}
 & Trial & 7 & 8 & 9 & 10 & 11 & 12  \\ \hline
  \parbox[t]{2mm}{\multirow{7}{*}{\rotatebox[origin=c]{90}{parameters}}} 
 & step size 	& 10  			& 10  			& 10 			& 10 					& 10 				& 10  \\
 & work avg 	& 60 			&  60 			& 60			& 60 					& 60 				& 60 \\
 & students 	& 1000 			& 1000 			& 1000 			& 2000 					& 2000 				& 2000 \\
 & places 		& 5 			& 5 			& 5 			& 10 					& 10 				& 10 \\
 & spaces 		& 250 			& 250 			& 250 			& 500 					& 500 				& 500 \\
 & function 	& most 			& prox 			& perc 			& most 					& prox 				& perc \\
 & trials 		& 10 			& 10 			& 10 			& 10 					& 10 				& 10 \\ \hline
\parbox[t]{2mm}{\multirow{3}{*}{\rotatebox[origin=c]{90}{results}}} 
 & ticks 		& $164.9\pm 14.8$ 	& $155.1\pm 21.7$ 	& $168.7\pm 15.4 $	& $153\pm 5.6 $	& $136\pm 3.7 $	& $154\pm 4.2$ \\
 & efficiency 	& 36.4\%			& 38.5\%			& 35.7\%				& 39.1\%				& 43.7\%				& 39.1\% \\
 & runtime (s) 	& 90.5			& 	93.8		& 	89.4			& 165.1			& 157.5			& 160.2
\end{tabular}
\caption{Parameters and results of trials continued \\ \footnotesize In results, tick value is ``$mean \pm margin \: of \: error$''}
\end{table}

% basic 1 and 2 for 20 trials

\begin{table}[]
\footnotesize
\centering
\begin{tabular}{l|l|ccc|ccc}
 & Trial & 13 & 14 & 15 & 16 & 17 & 18  \\ \hline
  \parbox[t]{2mm}{\multirow{7}{*}{\rotatebox[origin=c]{90}{parameters}}} 
 & step size 	& 10  			& 10  			& 10 			& 10 					& 10 				& 10  \\
 & work avg 	& 60 			&  60 			& 60			& 60 					& 60 				& 60 \\
 & students 	& 1000 			& 1000 			& 1000 			& 2000 					& 2000 				& 2000 \\
 & places 		& 5 			& 5 			& 5 			& 10 					& 10 				& 10 \\
 & spaces 		& 250 			& 250 			& 250 			& 500 					& 500 				& 500 \\
 & function 	& most 			& prox 			& perc 			& most 					& prox 				& perc \\
 & trials 		& 20 			& 20 			& 20 			& 20 					& 20 				& 20 \\ \hline
\parbox[t]{2mm}{\multirow{3}{*}{\rotatebox[origin=c]{90}{results}}} 
 & ticks 		& $159.5\pm 7.8$ 	& $147.2\pm 4.9$ 	& $159.3\pm 5.7 $	& $153.5\pm 3.20 $	& $137.3\pm 1.90 $	& $152.55\pm 2.53$ \\
 & efficiency 	& 37.7\%			& 40.7\%			& 37.8\%				& 39.1\%				& 43.8\%				& 39.4\% \\
 & runtime (s) 	& 82.2			& 69.0			& 86.2				& 182.0			& 151.3			& 187.2
\end{tabular}
\caption{Parameters and results of trials continued \\ \footnotesize In results, tick value is ``$mean \pm margin \: of \: error$''}
\end{table}

%Basic 3 and 4 for 20 trials

\begin{table}[]
\footnotesize
\centering
\begin{tabular}{l|l|ccc|ccc}
 & Trial & 19 & 20 & 21 & 22 & 23 & 24  \\ \hline
  \parbox[t]{2mm}{\multirow{7}{*}{\rotatebox[origin=c]{90}{parameters}}} 
 & step size 	& 10  			& 10  			& 10 			& 10 					& 10 				& 10  \\
 & work avg 	& 30 			&  30 			& 30			& 30 					& 30 				& 30 \\
 & students 	& 2000 			& 2000 			& 2000 			& 2000 					& 2000 				& 2000 \\
 & places 		& 10 			& 10 			& 10 			& 10 					& 10 				& 10 \\
 & spaces 		& 250 			& 250 			& 250 			& 100 					& 100 				& 100 \\
 & function 	& most 			& prox 			& perc 			& most 					& prox 				& perc \\
 & trials 		& 20 			& 20 			& 20 			& 20 					& 20 				& 20 \\ \hline
\parbox[t]{2mm}{\multirow{3}{*}{\rotatebox[origin=c]{90}{results}}} 
 & ticks 		& $114.7\pm 7.9$ 	& $88.4\pm 3.7 $ 	& $ 112\pm 6.5 $	& $193.6\pm 15.6 $	& $141.2\pm 6.3 $	& $199.4\pm 11.7 $ \\
 & efficiency 	& 26.1\%			& 33.9\%		& 26.8\%			& 15.4\%			& 21.1\%			& 15.0\% \\
 & runtime (s) 	& 112.5			& 89.9			& 119.4			& 186.0			& 119.9			& 194.1
\end{tabular}
\caption{Parameters and results of trials continued \\ \footnotesize In results, tick value is ``$mean \pm margin \: of \: error$''}
\end{table}

% runs 5 and 6

\begin{table}[]
\footnotesize
\centering
\begin{tabular}{l|l|ccc|ccc}
 & Trial & 25 & 26 & 27 & 28 & 29 & 30  \\ \hline
  \parbox[t]{2mm}{\multirow{7}{*}{\rotatebox[origin=c]{90}{parameters}}} 
 & step size 	& 10  			& 10  			& 10 			& 10 					& 10 				& 10  \\
 & work avg 	& 30 			&  30 			& 30			& 30 					& 30 				& 30 \\
 & students 	& 2000 			& 2000 			& 2000 			& 2000 					& 2000 				& 2000 \\
 & places 		& 20 			& 20 			& 20 			& 30 					& 30 				& 30 \\
 & spaces 		& 125 			& 125 			& 125 			& 83 					& 83 				& 83 \\
 & function 	& most 			& prox 			& perc 			& most 					& prox 				& perc \\
 & trials 		& 20 			& 20 			& 20 			& 20 					& 20 				& 20 \\ \hline
\parbox[t]{2mm}{\multirow{3}{*}{\rotatebox[origin=c]{90}{results}}} 
 & ticks 		& $138.8\pm 5.8$ 	& $92.3\pm 4.2 $ 	& $ 143.5\pm 6.2 $	& $175.3\pm 9.5 $	& $98.3\pm 5.9 $	& $177.4\pm 9.8 $ \\
 & efficiency 	& 21.5\%		& 32.4\%		& 21.0\%			& 17.1\%		& 30.3\%		& 16.9\% \\
 & runtime (s) 	& 142.1		&	93.5		& 	148.2		& 165.4			& 96.2			& 166.8
\end{tabular}
\caption{Parameters and results of trials continued \\ \footnotesize In results, tick value is ``$mean \pm margin \: of \: error$''}
\end{table}

\section{Analysis}



First, it cannot be concluded that information about the occupancy of a study location, like that provided by the UCL Go Study Space App, improves search efficiency. 

Second, in all trial sets, margins of error are unfortunately high. This was limited by computing resources as each table took between 30 and 60 minutes to compute. Still, some basic conclusions can be drawn as follows. 

Third, proximity was, on average, the most efficient  method in every trial set. 

Fourth, results for the $most$ and $perc$ search functions are very similar. Using $perc$ agents initially prioritize by the absolute value of study space in a patch because at $tick \: 0$ occupancy of all patches is 0.  The search functions diverge only after one space has been occupied. Thus the differences in outcomes are driven by the results for turtles who do not find space at the first patch searched. 

Fifth, comparing trial sets 19-21, 25-27, and 28-30 reveals the sensitivity of search time to the dispersion of study space. The total study space in a trial is the places parameter X spaces parameter. Thus the three sets of trials increase the number of places, and decrease the number of spaces, holding the total constant. Agents are on average, closer to study space anywhere in the environment, but search fewer spaces in a given tick when space is more distributed. It is clear that higher dispersion means less efficient search with ticks rising as dispersion increases. 

Sixth, comparing margins of error for trials 1-3 with 4-6, 7-9 with 10-12, and 13-15 with 16-18, more agents for the same trials reduces the margin of error. The ME's on the right are smaller than those on the left of each table. 

Finally, scaling the model up, moving closer to parameter values representative of UCL, increases the confidence allowed in the model outcome but overall results are consistent. Mean outcomes on the left and right sides of tables 2,3, and 4 display the same pattern, indicating that scaling the model down for computational reasons may not have had a negative effect on reliability. 

\section{Conclusions}

The key conclusion is that the UCL Study Space App may not have a material impact on search times for study space , whereas more tightly clustered study spaces may improve efficiency. AS search times for the newly built Student Center are better than for older places like the highly distributed space in the Main Library, the architecture of the university is improving the student experience. 

As mentioned above, a key unrealistic feature of this model is agents arriving simultaneously. This heavily emphasizes the search behavior at $tick \: 0$, which for $most$ and $perc$ are the same. A model that included stochastically determined arrival time may allow for greater differentiation between the two search functions. 

Finally, further work on the effect of scaling a model of this nature would be valuable to confirm conclusions drawn.  

\pagebreak

\printbibliography

\end{document}
